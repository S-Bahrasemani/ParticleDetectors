
\section*{Appendix}


%%%%%%%%%%%%%%%%%%%%%%%%%%%%%%%%%%%%%%%%%%%%
\subsection*{D$\emptyset$ Goals}
%%%%% SLIDE
\begin{frame}{\textcolor{Goldenrod}{D$\emptyset$ Experiment Goals}}

  \itt[<only@+>]
\item[$\Box$] \hlt{black}{Z and W masses:\\}
  $-$ Z mass from $ Z\to e^+ e^-$ using invariant mass\\
  $-$ W mass from $W\to l \nu_{l}$ using $m_{T} =
  \sqrt{2p^{l}_{T}\slashed{E_{T}}(1 - \cos\theta_{l, MET})}$\\
  $-$ good $\slashed{E_{T}}$ resolution is necessary for $m_T$
\item[$\Box$] \hlt{black}{Z and W widths:}\\
  $- $ \alert{$\delta \Gamma_Z = \frac{2}{N} \sqrt{\Gamma^2_Z + (2.35
      \delta m_Z)^2}$}\\
  $- $ $\frac{\Gamma_W}{\Gamma_Z}$ within 8\% $\to$ QCD radiative
  corrections  \& narrowing down $m_t$ mass window

\item[$\Box$] \hlt{black}{$Z \to X \gamma; X\to l^+l^-$ searches:} \\
  $- $ electronic to muonic decay ratio is needed\\
  $- $ \alert{ good $\gamma$ ( ~ 10 GeV) and $\pi^0$ separation is
    essential}
  
\item[$\Box$] \hlt{black}{ Forward-backward asymmetry:}\\
  $- $ $b\bar{b}$ decay was of special interest due to pure measurements
  from other experiments\\
  $- $ good muon identification at $p_T \approx 100 GeV/c$
  
%% 03_tri_couplings  
\item  \hlt{black}{Gauge boson couplings:\\}
  $- $ associated production of gauge bosons
  $- $ $p\bar{p} \to W \gamma X; W \to l \nu$ 

\item \hlt{black}{ W and Z production:}\\
  $- $ The x-dependence of W and Z production reveal the parton
  x-distribution in the same way as in Drell-Yan production.\\
  
  $- $ The $p_T$ distribution of produced bosons is interesting for a
  study of radiative processes in QCD

\item \hlt{black}{$W/Z \to q\bar{q}$:}\\
  $- $ good hadron energy measurement in order to reduce the error on jet
  invariant mass.\\
  $- $ good segmentation of calorimetry is also desired
  
\item \hlt{black}{ $\frac{\alpha_{QCD}}{\alpha_{QED}}$ :\\} 
  $- $ single $\gamma$ to single $g$ production 
  \tti
  
\end{frame}


%%%%%%%% SLIDE
\begin{frame}{\textcolor{Goldenrod}{D$\emptyset$ Design Considerations}}
  \itt
\item[$\bullet$]
  \textcolor{blue}{Electromagnetic energy resolution at the level of $\delta E / E
    = \frac{0.05}{\sqrt{E}}$ with good $\pi^0-e $ separation}
  {\small
    \itt
  \item good electron ID for narrow massive states searches and
    to lower electron-jet faking rate. \\
  \item good lateral and longitudinal sampling for single $\gamma$
    searches with high $p_T$ (\alert{challenging $\pi^0-e $ separation})
    \tti
  }
  
\item[$\bullet$]
  \textcolor{blue}{good muon momentum resolution and
    muon ID}
  {\small
    \itt
  \item muons are less analyzed, but with charge tagging $\to$ ID them even inside
    jets which is important for many searches
    \tti
  }
  \tti
\end{frame}


%%%%%%% SLIDE 
\begin{frame}{\textcolor{Goldenrod}{Upgraded D$\emptyset$ }}
  % \<{0.5\textwidth}
  % \begin{center}
  %   \img{01_dzero_wholedetector}
  % \end{center}
  % % \caption*{{\scriptsize A simplified cross section view of the
  % % D$\emptyset$   }}
  % \>
  \itt
\item[$\Box$]<only@1> \hlt{Blue}{in 2001 after the Main Injector and associated Tevatron
  upgrades the instantaneous luminosity increased by more than a factor of ten}
  
\item[$\Box$]<only@1> \hlt{Magenta}{The central tracking system was completely replaced.}
  {\small
    \itt
  \item \alert{the old system lacked a magnetic field and suffered from radiation
      damage}
  \item improved tracking technologies are now available.\\
  \item The new system included a silicon microstrip tracker and a
    scintillating-fiber tracker located within a 2 T solenoidal
    magnet.
    \tti
  }
\item[$\Box$]<only@2> \hlt{blue}{Between the solenoidal magnet and the central calorimeter and
    in front of the forward calorimeters, preshower detectors have
    been added for improved electron identification.}
  
\item[$\Box$]<only@2> \hlt{blue}{In the forward muon system, proportional drift chambers are
    replaced by mini drift tubes and trigger scintillation
    counters\\}
  {\small
    \itt
  \item which can withstand the harsh radiation environment and
    additional shielding has been added.
    
  \item \alert{In the central region,
      scintillation counters have been added for improved muon
      triggering.}
    \tti
  }
  \tti
\end{frame}


%%%%%%% SLIDE
\begin{frame}{\textcolor{Goldenrod}{p-n Junction}}
  \begin{figure}[h]\centering
    \includegraphics[width=0.95\linewidth]{./Images/104_extra_pn_junction}
  \end{figure}
\end{frame}

%%%%%%% SLIDE
\begin{frame}{\textcolor{Goldenrod}{Basic Silicon Detector}}
  \begin{figure}[h]\centering
    \includegraphics[width=0.95\linewidth]{./Images/105_extra_silicon_detector}
  \end{figure}
\end{frame}


%%%%%%% SLIDE
\begin{frame}{\textcolor{Goldenrod}{DSDM Silicon Detector}}
  \begin{figure}[h]\centering
    \includegraphics[width=0.95\linewidth]{./Images/106_extra_DSDM.pdf}
  \end{figure}
\end{frame}


%%%%%%% SLIDE
\begin{frame}{\textcolor{Goldenrod}{Visible Light Photon Counters}}
  \begin{figure}[h]\centering
    \includegraphics[width=0.95\linewidth]{./Images/107_extra_VLPCs}
  \end{figure}
\end{frame}


%%%%%% SLIDE
\begin{frame}{\textcolor{Goldenrod}{Central Track Tigger}}
  \(
  \<{0.6\textwidth}
  \itt
\item Counts track candidates identified in axial view of CFT by
  looking for hits in all 8 axial layers
\item Combines tracking and preshower information to identify
  electron and photon candidates
\item
  Generates track lists allowing other trigger systems to
  perform track matching
  \tti
  \>
  \<{0.4\textwidth}
  \img{20_CFT}
  \>
  \)
\end{frame}



%%%%%%%%%%%%%%%%%%%%%%%%%%%%%%%%%%%%%%%%%%%%
\subsection*{Solenoidal magnet}
%%%%%% SLIDE
\begin{frame}{\textcolor{Goldenrod}{Solenoidal Magnet }}
  \(
  \<{0.45\textwidth}
  \img{26_magnet_solonoid.pdf}
  \>
  \<{0.7\textwidth}
  \itt
\item[$\Box$]<1-> to optimize the momentum resolution, $\delta p_T /p_T$ and tracking
  pattern recognition $\to $ a central field of $2 T$
\item[$\Box$]<2-> design criteria:
  \itt
\item [$i)$] to operate safely and stably at either polarity
\item [$ii)$] a uniform field over as large a percentage of the volume as practical,
\item [$iii)$] as thin as possible to make the tracking volume as large as possible,
\item [$iv)$] an overall thickness of approximately $1 X_0$ at $\eta = 0$ to optimize
  the performance of the central preshower detector mounted on the outside of
  the solenoid cryostat.
  \tti
  \tti
  \>
  \)
\end{frame}

%%%%%% SLIDE
\begin{frame}{\textcolor{Goldenrod}{Solenoidal Magnet }}
    \begin{figure}[h]
      \centering
      \includegraphics[height=0.6\textheight]{./Images/25_magnet_solonoid.pdf}
      \caption*{The solenoid is wound with two layers of superconductor to achieve the required
        linear current density for a $2 T$ central field.}
    \end{figure}
    
\end{frame}

%%%%%% SLIDE
\begin{frame}{\textcolor{Goldenrod}{Magnet Field}}
  \(
  \<{0.55\textwidth}
  \itt[<+->]
\item Within the solenoid (operated
  at $4749 A$), The calculated magnetic field is scaled by 0.09\%
  to agree with the measurement.
\item Within the toroid (operated at $1500 A$)
  The calculated magnetic field is scaled by $4.3\%$
  \tti
  \>
  \<{0.6\textwidth}
  \img{28_magnet_solonoid.pdf}\\
  {\scriptsize The $y-z$ view of the $D\emptyset$ magnetic field (in
    $kG$). The field
    in the central toroid is approximately $1.8 T$}
  \note{with both the toroidal
    and solenoidal magnets at full current ($1500 A$ and $4749 A$, respectively).}
  \>
  \)
\end{frame}


%%%%%%%%%%%%%%%%%%%%%%%%%%%%%%%%%%%%%%%%%%%%%%
\subsection*{Muon detector}
%%%%%% SLIDE
\begin{frame}{\textcolor{Goldenrod}{Muon Tracking System }}
  \begin{overlayarea}{\textwidth}{\textheight}
    \begin{figure}[h]
      \centering
      \includegraphics[height=0.7\textheight]{./Images/42_MD_central_magnet}
      \caption*{{\scriptsize Magnetic field in the central toroid
          magnet. The magnetic field is in $kG$.}}
    \end{figure}

  \end{overlayarea}
\end{frame}


%%%%%% SLIDE
\begin{frame}{\textcolor{Goldenrod}{Combinatorial Background}}
  \begin{overlayarea}{\textwidth}{\textheight}
    \begin{figure}[h]
      \centering
      \includegraphics[height=0.5\textheight, width=0.5\textwidth]{./Images/103_extra_combinatorial_bkgs_01}
      \includegraphics[height=0.5\textheight, width=0.5\textwidth]{./Images/103_extra_combinatorial_bkgs_02}
    \end{figure}
    
    \itt
  \item the non-background background!
    \tti
  \end{overlayarea}
\end{frame}


