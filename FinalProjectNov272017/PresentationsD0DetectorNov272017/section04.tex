

\section{Preshower detectors}

\subsection{Central preshower detector}
%%%%%% SLIDE
\begin{frame}{\textcolor{Goldenrod}{Preshower detector }}
  \begin{overlayarea}{\textwidth}{\textheight}
    \(
    \<{0.45\textwidth}
    \itt[<+->]
  \item calorimetery as well as tracking detectors:\\
    electron ID and background rejection at online triggering and
    offline reconstruction.
  \item triangular strips of scintillator with a wavelength-shifter at
    center of each strip and a waveguide transfering light to PMTs.
  \item The central preshower detector (CPS) consisted of three
    concentric cylindrical layers ($|\eta| < 1.3$)
  %   % and is located between the solenoid and the central calorimeter
  % \item The two forward preshower detectors (FPS) cover $1.5 < |\eta| < 2.5$
  %   % and are attached to the faces of the end calorimeters.
  % \item a wavelength-shifter at center of each strip
    \tti
    
    \note{, enhancing the
      spatial matching be- tween tracks and calorimeter showers [78]. The
      detectors can be used offline to correct the electromagnetic energy
      measurement of the central and end calorimeters for losses in the
      solenoid and upstream material}
    
    \note{Since the triangles are
      interleaved, there is no dead space between strips and most tracks
      traverse more than one strip, allowing for strip-to-strip
      interpolations and improved position measurement.}

    \note{The preshower detectors share common elements with the central
      fiber tracker, beginning with the waveguides and continuing
      through the entire readout elec- tronics system.}
    \>

    \<{0.7\textwidth}
    \img{29_PS_forward.pdf}
      %\img{33_PS}\\
      %\img{31_PS}\\
    \>

    \)
  \end{overlayarea}
\end{frame}


\subsection{Forward preshower detector}
%%%%%% SLIDE
\begin{frame}{\textcolor{Goldenrod}{Forward Preshower detector }}
  \(
  \<{0.75\textwidth}
  \itt[<+->]
\item[$\bullet$] The upstream layers are known as the minimum ionizing particle,
  or MIP, layers while the downstream layers behind the absorber are
  called the shower layers.
\item[$\bullet$] Charged particles passing through the detector will register
  minimum ionizing signals in the MIP layer $\to$ tracking.
  
\item[$\bullet$] {\small Electrons shower in the absorber $\to$ a cluster of
  energy,
  % (typically on the order of three strips wide),
  which is then matched
  to MIP-layer signal.}
  
\item[$\bullet$] {\small Photons will not generally interact in the MIP layer, but will
  produce a shower signal in the shower layer.}
  
\item[$\bullet$] {\small Heavier charged particles are less likely to shower, typically
  producing a second MIP signal in the shower layer.}
  \tti
  \>
  \<{0.4\textwidth}
  \img{30_PS}\\
  {\scriptsize FPS module with  $u -v$ MIP
    and shower layers, separated by a lead and stainless steel absorber.}
  \>
  \)
\end{frame}

